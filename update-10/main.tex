\documentclass{article}
\usepackage[utf8]{inputenc}

\title{Update-10}
\author{Asim Anand}
\date{March 2022}

\begin{document}

\maketitle

\section{Nuclear Medicine in India}
Rectilinear Scanner was first installed at INMAS in 1962–1963. Dr. R. S. Satosker undertook the first thyroid uptake measurement study using Thyroid Probe in 1960 at KEM Hospital, Mumbai. Later, standardization of uptake measurements was done at RMC. RMC mounted thyroid uptake probe in 1965. RMC procured slow rectilinear scanner in 1965, which was replaced by the fast rectilinear scanner in 1969. Dr. Aban M Samuel developed the country's first RIA laboratory at RMC in 1967. RMC has the distinction of acquiring India's first scintillation gamma camera in 1969. Single-photon emission computed tomography (SPECT) gamma camera was added to RMC armamentarium in 1984. INMAS was rather late in getting a gamma camera, which became available in 1982. India's first magnetic resonance imaging (MRI) scanner was commissioned at INMAS, Delhi in 1986. In October 2002, the then Prime Minister of India, late Shri Atal Behari Vajpayee inaugurated the country's first medical cyclotron and positron emission tomography (PET) scanner at RMC. India's first dedicated PET-CT scanner was commissioned in the Department of Molecular Imaging of TMH, Mumbai in December 2004. The installation of the first cyclotron (2002) in the basement of annexe building and first PET-CT (2004) at the main building of TMH (2004) acted as a trigger for a virtual revolution of molecular (PET) imaging in the country. As per the recent industry report, 222 PET-CT scanners and 19 cyclotrons have already been installed in India till 2018. There are a total of 293 nuclear medicine departments in India as per the Atomic Energy Regulatory Board's (AERB) recently released list (July 2018). Today, 22 of 29 states (76percent) and 3 of 7 union territories (43 percent) in India have nuclear medicine facility. India estimated to have one PET-CT, one gamma camera and one nuclear medicine department per 5 million population. However, it is an astonishing fact that nuclear medicine was established first in the government institutions, but there are merely 14percent of nuclear medicine departments currently functioning in the government sector, may be because primary healthcare has been the priority of government. After installation of the first gamma camera in 1969, <20 cameras existed in 1985, 64 gamma cameras until 1990, which have now grown to 233 in number including SPECT-computed tomography (SPECT-CT).
\end{document}
