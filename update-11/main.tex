\documentclass{article}
\usepackage[utf8]{inputenc}

\title{Update-11}
\author{Asim Anand}
\date{April 2022}

\begin{document}

\maketitle

\section{Radiation Technologies and Applications in India}
India is a large producer of radioisotopes. The radioisotopes are produced in the research reactors at Trombay, accelerator at Kolkata and various nuclear power plants. BARC, BRIT, CAT and VECC are the organizations of DAE which are engaged in the development of radiation technologies and their applications in the areas of health, agriculture, industry and research.

DAE is working in close co-operation with other organizations of the Government of India to widen the reach of these technologies for the benefit of the common man. Remarkable progress was achieved in applications of Radioisotopes and Radiation Technology in the areas of nuclear agriculture, food preservation and industry.

Research Reactors:

The research reactors APSARA, CIRUS and DHRUVA at Trombay are utilized for basic and applied research, isotope production, material testing and training for human resource development.

During the report period, APSARA and DHRUVA reactors operated satisfactorily and were extensively used for basic and applied research, radioisotope production, material testing and operator training. After refurbishing, CIRUS went into operation at 20 MWt. The work on the 20 MWt light water cooled, heavy water reflected low enriched uranium fuelled pool type research reactor made progress.


Radionucleids Production:

At Trombay, 72 TBq of activity of which Molybdenum-99 and Iodine-131 constitute the bulk was processed and supplied through BRIT to different end users. Radionuclides such as Phosphorus-32, Samaraium-153, Mercury-203, Holmium-166, Bromine-82, and Gold-198 were regularly processed and supplied. Iodine-125 was produced by irradiating natural Xe gas for further preparation of Iodine-125 based radio-pharmaceuticals and sources.
\end{document}
