\documentclass{article}
\usepackage[utf8]{inputenc}

\title{Term Paper update 1}
\author{asim anand}
\date{January 2022}

\begin{document}

\maketitle

\section{A General Introduction on Radiopharmaceuticals and Nuclear Medicine}
Radiopharmaceuticals are radioactive compounds which  have  a  bound  radionuclide  in  their  structure,  whose purpose is directing the radionuclide to a location to be treated or to obtain images. Nuclear medicine is the medical specialty that employs radiopharmaceuticals, which has presented itself as a tremendously  useful  ally  for  medicine  assisting  in  various diagnoses  and  treatments,  especially  for  cancer.  The  general objective of this work is to identify the main radionuclides and metal  complexes  currently  used  as  radiopharmaceuticals.  The main  metal  complexes  used  as  radiopharmaceuticals  are compounds of technetium (99mTc) like sodium pertechnetate and methylenediphosphonate  MDP-99mTc and  other  compounds  of indium (111In), thallium (201Tl), gallium (67Ga, 68Ga), iodine (123I and  131I),  chromium  (51Cr),  sulphur  (35S),  phosphorus  (32P), fluorine (as  fluorodeoxyglucose, 18F-FDG and  sodium fluorine, Na18F),  which  are  widely  used  in  the  nuclear  medicine  for diagnosis by imaging. They have been of great importance for the early diagnosis of numerous diseases, mainly cancer.Currently, technetium compounds are the majority of radiopharmaceuticals used in all  countries.  In  Brazil,  Institute  of  Energy  and  Nuclear  Research  (IPEN)  is  one  of  the  most  important  distributors  of radiopharmaceuticals, producing, importing and distributing them to clinics and hospitals over the country
\end{document}
