\documentclass{article}
\usepackage[utf8]{inputenc}

\title{Term Paper update-3}
\author{Asim Anand}
\date{4th February 2022}

\begin{document}

\maketitle

\section{Radiopharmaceuticals}
In nuclear medicine, radiopharmaceuticals are used in diagnostic imaging and radiotherapy, being of  utmost importance for  medicine in general to assist  in  diagnoses  of  organs  and  treatments  of pathological conditions, especially cancer.  In the imaging  modality,  radiopharmaceuticals  are administered via oral, intravenous, or by inhalation to  enable  visualization  with  their  radioactive tracers of various organs, such as kidneys, lungs, thyroid and heart functions, bone metabolism and blood circulation. In therapeutic modality, aiming to treat cancer or over functioning thyroid gland, a high dose of radiation is delivered through specific radiopharmaceuticals  targeting  the  diseased organ.

Radiopharmaceuticals generally consist of two components, a radioactive element (radionuclide), that  permits  external  scan,  linked  to  a  non-radioactive  element,  a  biologically  active molecule, drug or cell (red and white blood cells labeled with a radionuclide, for example) that acts as a carrier or ligand, responsible for conducting the radionuclide to a specific organ.

Some  characteristics  are  necessary  for considering radiopharmaceuticals clinically useful for imaging: the decay of the radionuclide should be in specific ranges of energy emissions (511 keV for positron emission tomography – PET and 100-200  keV  for  gamma  cameras)  and  in  sufficient quantity for tomography detection; 2) it should not contain  particulate  radiation  (beta emissions,  for example), because it increases the radiation dose in patients; 3) the half-life should be for a few hours only;  4)  the  radionuclides  should  not  be contaminated by other radionuclides of the same element nor even its stable radionuclides (carrier-free); 5) they should have specific activity, and the highest  specific activity  comes  from carrier-free radionuclides;6)  the radiopharmaceutical  should not  have  toxicity  and  does  not  manifest physiological effects;  7) the  radiopharmaceutical should be available for instant usage and easy to compound;  8)  the  radiopharmaceutical  should reach  the  target  organ  quickly  and  accurately, according to its intended application. 

Diagnostic  radiopharmaceuticals  have  no pharmacological effects and their administration is not associated with relevant clinical side effects. Its clinical use, however, carries the inherent risk of exposure to radiation and possible contamination during  radiopharmaceutical  formulation,  since most  radiopharmaceuticals  are  administered intravenously.
\end{document}
