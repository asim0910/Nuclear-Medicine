\documentclass{article}
\usepackage[utf8]{inputenc}

\title{Term Paper update-8}
\author{Asim Anand}
\usepackage{graphicx}
\date{March 2022}

\begin{document}

\maketitle

\section{Radiopharmaceuticals for diagnosis in human body }
Medical doctors and chemists have identified a large  number of  chemicals that  are absorbed  by specific  organs.  Thyroid,  for  example,  absorbs iodine while the brain absorbs glucose. Diagnostic radiopharmaceuticals can be used to monitor blood flow to the brain, liver, lung, heart, and kidney. Particulate  radiation  can  be  useful  for destroying  or  weakening  cancer  cells (radiotherapy). The radionuclide that generates the radiation can be located in a certain organ in the same way used for diagnostics. In many cases, beta radiation  causes  the  destruction  of  cancer  cells. 177Lutetium (177Lu), for example, is prepared from 176ytterbium  (176Yb)  which  is  irradiated  to transform  it  into  177Yb, which rapidly  returns to 177Lu.  90Yttrium  (90Y)  is  used  to  treat  cancer, especially  non-Hodgkin’s  lymphoma  and  liver cancer.  131Iodine  (131I),  153samarium  (153Sm)  and 32phosphorus (32P) are  also used in  radiotherapy. 131Cesium  (131Cs),  103palladium  (103Pd)  and 223radium (223Ra) are used in special cases.


Figure 1 lists the radionuclides most commonly used for diagnosis and treatment of different organs of the human body. 


\begin{figure}[htp]
    \centering
    \includegraphics[width=4cm]{pic}
    \caption{Image}
    \label{fig:galaxy}
\end{figure}

\end{document}
