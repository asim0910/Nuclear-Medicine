\documentclass{article}
\usepackage[utf8]{inputenc}

\title{Term Paper update-5}
\author{Asim Anand}
\date{February 2022}

\begin{document}

\maketitle

\section{Nuclear Medicine Techniques}
Diagnostic techniques in nuclear medicine use radioactive tracers that emit gamma radiation from within the body. The camera constructs an image from the points where the radiation is emitted. This image is magnified on a computer  and it can be observed  on  a  monitor  that  indicates  the anomalies.

The nuclear medicine techniques include Single Photon  Emission  Computerized  Tomography (SPECT), Positron Emission Tomography (PET), and  computed  tomography-PET  (PET-CT)  (for better anatomical visualization), micro-PET (with ultra-high resolution) and microcomputerized axial tomography  micro-CAT.  These techniques  are used to analyze biochemical dysfunctions as early signs  of  the disease,  its  mechanisms  and association  with  disease  states  from  cancer  to cardiovascular diseases and mental disorders.

A SPECT exam is  used primarily to visualize the blood flow through veins and arteries, and to perform  pre-surgical  evaluation  of  seizures. SPECT  is also useful  in the  diagnosis of  blood deprived  areas of  brain  (ischemic),  spinal  stress fractures (spondylolysis) and tumors.

The PET  imaging detects  the pair of  gamma rays produced by the interaction between a positron and  an electron  in  the tissues  of  the body.  The electron and  the  positron  neutralize  each  other producing two gamma rays in opposite directions. PET detects the electronic signal converting with scintillation crystals the energy released by gamma rays.

99mTechnetium is the radionuclide that has the best  characteristics  to  combine  with  gamma cameras  and  18fluorine  has  the  most  desirable characteristics for PET.

Although  the  SPECT  and  PET  techniques capture images with high intensity, they have low spatial resolution because they are directed to the surface of the body they are accurately visualized. On the other hand, computerized tomography (CT) and  magnetic  resonance  have  greater  spatial resolution, but with less sensitivity. To obtain such restrictions, the techniques are merged for images with  excellent  spatial  resolution  combined  with high sensitivity.

X-ray  CT  has  a  computational  process  that makes  a  three-dimensional  image,  resulting  in images with much greater resolution and quantity of details of internal structures and organs of the body6.  There  are  other  non-nuclear  techniques, which are not described in this review, but could be accessed in this reference.
\end{document}
