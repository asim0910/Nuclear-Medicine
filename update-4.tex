\documentclass{article}
\usepackage[utf8]{inputenc}

\title{Term Paper update-4}
\author{Asim Anand}
\date{February 2022}

\begin{document}

\maketitle

\section{Historical base of Nuclear Medicine and Radioactivity}
Wilhelm Roentgen’s work on X-ray studies has stimulated  researchers  such  as  Henri  Poincaré whose studies are related to the hypothesis of X-ray emission  and fluorescence.  The  first  scientist  to carry out the hypotheses proposed by Poincaré was Charles  Henry,  using  zinc  sulfide  as  an  X-ray intensifier, concluding that in the presence of light, when the radiographs became sharper because of the substance.

In 1896, Henri Becquerel used uranium salts on photographic  plates,  which  resulted  in  marked radiographs without the presence of light. In 1905, Marie and  Pierre Curie  were the  first to suggest radium for treatment of cancer. The Curies’ work may  be  considered  the  beginning  of  modern nuclear medicine. In 1931, Ernest Lawrence built the  first  cyclotron,  equipment  that  accelerated alpha  particles,  such  as  protons,  deuterons,  or helium  ions,  with  the  aim  of  penetrating  the nucleus to produce stable and radioactive isotopes. A decade later, Lawrence’s cyclotron had produced 223 radioactive isotopes, many of which are now of great value for medicinal uses and studies in the biological sciences.

In  1930,  Ernest  Lawrence  and  Milton Livingstone, with their invention of the cyclotron, allowed the artificial production of new radioactive elements, but the quantities were very small. The medical use of radionuclides began during World War II with the Oak Ridge reactor in the United States, initiating the production of radionuclides in global scale.  Hal Anger,  in 1958,  developed  the image-scintillation chamber, which did not require the  movement  of  the  detector.  It  had  a  higher geometric resolution, and it was possible to obtain different projections of the same distribution of the radiopharmaceutical.  However,  computers  were not yet capable of acquiring the information and transforming it  into  images. So,  the information was sent to the cathode ray tube for it to be recorded on  photographic  plates  or  films.  The  modern Short review scintillation cameras used nowadays are the Anger camera type.

Nuclear Medicine only had a diagnostic power when  Paul  Harper and  his group  introduced  the 99mTc radionuclide as a marker. This radionuclide decays by isometric transition emitting photon with energy  of  140  keV,  gamma-type  radiation  and physical half-life of about 6 hours, which allows studies with reasonable intervals. In addition, it is obtained by the decay of the parent element 99Mo, produced in 99Mo/99mTc generators.

The  first  radiopharmaceuticals  were commercialized  in  1950.  131Iodine  was  the  first commercially  available  isotope,  with  Abbott Laboratories being  the first company to  produce radiopharmaceuticals for medical uses.

The radioactive elements, thus classified, may have highly energetic unstable nuclides due to the excess of energy, which stabilizes by the emission of particles or electromagnetic radiation or charged particles during radioactive decay. In this context, there are three types of radiation: alpha, beta minus and  gamma1,2. Radiation  propagates at  a  certain speed  and  contains  energy  with  electric  and magnetic charges that can be generated by natural sources or by artificial devices, such as a cyclotron. Ionizing  radiation  is  generated  from  the  energy emitted by an unstable nucleus in artificial form or by a cyclotron.
\end{document}
