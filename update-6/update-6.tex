\documentclass{article}
\usepackage[utf8]{inputenc}

\title{Term Paper Update-6}
\author{Asim Anand}
\date{February 2022}

\begin{document}

\maketitle

\section{Radionuclides in medicine}
Radionuclides  have  many  applications  in several areas that use nuclear technology. The use of  radiation  and  radionuclides  in  medicine  is continuously  increasing  both  for  diagnosis  and therapy worldwide.

In  developed  countries  (1/4  of  the  world’s population), one person in 50 is subject to nuclear medicine and the frequency of radionuclide therapy is  about  10 percent  of  that  number,  according  to  the World Nuclear Association.

Radiation is used in nuclear medicine to obtain information  about  the  organs  of  a  person  for treatment of a disease. In many cases, information is used for a quick diagnosis. Thyroid, bones, heart, liver, kidney and many other organs can be easily observed in the generated image and the anomalies of  its  functions  are  revealed.  About  10,000 hospitals worldwide use  radionuclides and about 90 percent  of  the  procedures  are  for  diagnosis.  The radionuclide most used in diagnostics is 99mTc. It has been used in about 40 million exams per year, which means 80 percent of all exams in nuclear medicine worldwide.

Radionuclides  are  essential  components  of diagnostic  exams.  In  combination  with  the equipment recording the images from the emitted gamma rays,  the processes that occur  in various parts of the body can be studied. For diagnosis, a dose  of  the radioactive  material  is given  to  the patient  and the  localization in  the organ  can  be studied  as  a  two-dimensional  image  or,  using tomography, as a three-dimensional image. These gamma  or  positron  tracers  have  short-lived isotopes and are linked to chemical compounds that allow  specific  physiological  processes  to  be evaluated.

In  the  USA,  more  than  20  million  medical applications  per  year  are  performed  using radionuclides and in Europe about 10 million per year11. In Brazil, the Nuclear and Energy Research Institute (IPEN) reported that, in 2017, there were 360  diagnostic  clinics  and  nuclear  medicine hospitals, 70 percent in the South and Southeast regions of Brazil, 72 PETs installed, others to be licensed, 33  hospitals  with  rooms  for  therapy  and approximately 1.8 million patients per year.


\end{document}
