\documentclass{article}
\usepackage[utf8]{inputenc}

\title{Update-9}
\author{Asim Anand}
\usepackage{graphicx}
\date{March 2022}

\begin{document}

\maketitle

\section{Production of radionuclides}
Radionuclides found in nature, such as uranium and radium, are heavy elements with high toxicity and long half-life (over 1,000 years), so they are not used clinically. Radionuclides used in nuclear medicine  are  artificially  produced  by  neutron bombardment or nuclear fission. 

Many  radionuclides  are  produced  in  nuclear reactors  and  cyclotrons.  Generally,  neutron-rich radioisotopes  and  those  resulting  from  nuclear fission are produced in reactors (Table 1) and the neutron-poor  radioisotopes  are  produced  in cyclotrons  (Table  2).  There  are  about  30 radioisotopes  produced  by  activation  and  5  are reactor melt products. A list with 70 elements, their isotopes,  half-lives,  decay,  main  energy  and applications can be found in the reference. 

\begin{figure}[htp]
    \centering
    \includegraphics[width=4cm]{pic}
    \caption{Image}
    \label{fig:galaxy}
\end{figure}

\end{document}
